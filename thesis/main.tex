\begin{filecontents*}{\jobname.xmpdata}
  \Title{Building a Ray-Tracing Engine on Sparse Voxel Grids}
  \Author{10834225}
  \Language{en-GB}
  \Copyrighted{True}
\end{filecontents*}

\documentclass{extra}

%%%%%%%%%%%%%%%%%% PACKAGES AND COMMANDS %%%%%%%%%%%%%%%%%%
\usepackage{graphicx,psfrag,color} % for postscript graphics files
  \graphicspath{ {./figures/} }
\usepackage{amsmath}               % assumes amsmath package installed
  \allowdisplaybreaks[1]           % allow eqnarrays to break across pages
\usepackage{amssymb}               % assumes amsmath package installed
\usepackage{url}                   % format hyperlinks correctly
\usepackage{rotating}              % allow portrait figures and tables
\usepackage{multirow}              % allows merging of rows in tables
\usepackage{lscape}                % allows pages to be typeset in landscape mode
\usepackage{tabularx}              % allows fixed width tables
\usepackage{verbatim}              % enhanced version of built-in verbatim environment
\usepackage{footnote}              % allows more control over footnote environments
\usepackage{float}                 % allows H option on floats to force here placement
\usepackage{booktabs}              % improve table line spacing
\usepackage{lipsum}                % for adding dummy text here
\usepackage[base]{babel}           % for proper hypthenation in lipsum sections
\usepackage{subcaption}            % for multiple sub-figures in a single float
\usepackage{cleveref}
\newtheorem{theorem}{Theorem}

\newcommand{\degree}{\ensuremath{^\circ}}
\newcommand{\sus}[1]{$^{\mbox{\scriptsize #1}}$} % superscript in text (e.g. 1st)
\newcommand{\sub}[1]{$_{\mbox{\scriptsize #1}}$} % subscript in text

%%%%%%%%%%%%%%%%%% REFERENCES SETUP %%%%%%%%%%%%%%%%%%
\usepackage[style=ieee,backend=biber,backref=true,hyperref=auto,backend=bibtex]{biblatex}
% \DefineBibliographyStrings{english}{backrefpage = {cited on p\adddot},  backrefpages = {cited on pp\adddot}}
\addbibresource{ref.bib}

%%%%%%%%%%%%%%%%%% START DOCUMENT %%%%%%%%%%%%%%%%%%
\begin{document}
\makeatletter
\title{\xmp@Title}
\studentid{\xmp@Author}
\makeatother

\course{Computer Science}
\submitdate{2024}
\wordcount{many}
\maketitle



%%%%%%%%%%%%%%%%%% LISTS OF CONTENT %%%%%%%%%%%%%%%%%%
\uomtoc
% other lists are not required, but can include \uomlof and \uomlot if really want to


%%%%%%%%%%%%%%%%%% ABSTRACT %%%%%%%%%%%%%%%%%%
\begin{abstract} % put abstract here. Limit is 1 page.
  This is abstract text.

  \lipsum[1-2]
\end{abstract}%
\clearpage


%%%%%%%%%%%%%%%%%% DECLARATIONS %%%%%%%%%%%%%%%%%%
\uomdeclarations % Don't need unless final thesis


\begin{uomacknowledgements}
What should I aknowledge?! I am citing \cite{this}, and \cite{that}
\end{uomacknowledgements}

%%%%%%%%%%%%%%%%%% CONTENT %%%%%%%%%%%%%%%%%%
%%% Local Variables:
%%% mode: latex
%%% TeX-master: "../main"
%%% End:

\part{Introduction}
\section{Motivation}
\section{Aims}
\section{Objectives}
\section{Report structure}


%%%%%%%%%%%%%%%%%% REFERRENCES %%%%%%%%%%%%%%%%%%
\printbibliography[title={References},heading=bibintoc]

%%%%%%%%%%%%%%%%%% APPENDICES %%%%%%%%%%%%%%%%%%
\begin{uomappendix}
    \section{Project outline}
    Project outline as submitted at the start of the project is a required appendix.

    \section{Risk assessment}
    Risk assessment is a required appendix. Put here. And there as well
\end{uomappendix}

\end{document}
