%%% Local Variables:
%%% mode: latex
%%% TeX-master: "../main"
%%% End:

\part{Introduction}\label{intro}
\section{Motivation}
The field of computer graphics has witnessed substantial evolution, driven by countless advancements in technology and an ever-growing demand for more realistic and interactive digital experiences. Rendering engines, fundamental to this progression, are the drivers behind compelling visual content in various applications, ranging from cinematic visual effects to complex scientific visualisations and immersive video games.

Over the last decade, advancements in hardware, particularly GPUs, have significantly expanded the capabilities of graphical applications. Modern GPUs offer tremendous computational power, enabling more complex calculations at higher speeds. This hardware evolution has made real-time rendering, particularly ray tracing, more feasible for widespread use. Thanks to these advancements, Ray tracing, once limited to pre-rendered scenes due to its computational intensity, can now be performed in real time. This shift has revolutionised gaming and interactive media, allowing for cinematic-quality graphics during gameplay.

This project aims to push the boundaries of what can be achieved with voxelised real-time rendering by leveraging efficient sparse voxel data structures. The goal is to develop a tool that supports the creation of visually interesting digital environments and contributes to the ongoing research and development in rendering technologies.

This project, therefore, stands at the intersection of theoretical exploration and practical application. It aims to harness the power of modern hardware to solve complex rendering challenges and contribute valuable insights and tools to the field of computer graphics.

\section{Objectives}
\label{obj}
\begin{enumerate}
  \item Develop advanced ray tracing algorithms that fully utilise modern hardware.
  \item Explore acceleration structures that optimise ray casting performance.
  \item Develop a voxel rendering engine that integrates these advanced algorithms and structures.
  \item Design the engine architecture to take full advantage of the algorithms and technology it is built on, ensuring robustness, efficiency, and rendering performance.
  \item Compare and test these algorithms against each other to validate improvements in speed and quality.
\end{enumerate}

\section{Aims}
\label{aims}
\begin{enumerate}
\item \emph{Performance:} Optimize the rendering engine to handle complex scenes with high levels of detail and dynamic changes efficiently, striving for speed and graphical output enhancements.
  \item \emph{Safety:} Make sure the system is reliable, minimize memory leaks and undefined behaviours
  \item \emph{Cross-Platform:} Ensure the engine is not tied to a specific platform, operating system or graphics backend
  \item \emph{Futre-Proofing:} Build the engine on a forward-looking graphics API designed to be efficient, powerful, and broadly supported.
\end{enumerate}

\section{Report structure}
The report is comprised of 5 parts:

\begin{itemize}
    \item \Cref{intro} gives an introduction to the project and the report
    \item \Cref{backgorund} details the project background, literature review and related work
    \item \Cref{methodology} presents a detailed walkthrough of the design and development of the project
    \item \Cref{results} shows the results and experiments of the project
    \item \Cref{conclusions} presents the project conclusions
\end{itemize}
