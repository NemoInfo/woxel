%%% Local Variables:
%%% mode: latex
%%% TeX-master: "../main"
%%% End:

\part{Introduction}\label{intro}
\section{Motivation}



\section{Objectives}
\label{obj}
\begin{enumerate}
  \item Develop advanced ray tracing algorithms that take full advantage of modern hardware.
  \item Explore accelleration structures that optimize ray casting performance.
  \item Devolp a voxel rendering engine that integrates these advanced algorithms and structures.
  \item Desgin the engine architecture to take full advantage of the algorithms and technology it is built on, ensuring robustness, efficiency, and rendering performance.
  \item Compare and test these algorithms against eachother to validate improvements in speed and quality.
\end{enumerate}

\section{Aims}
\label{aims}
\begin{enumerate}
  \item \emph{Performance:} Optimize the rendering engine to handle complex scenes with high levels of detail and dynamic changes efficiently, striving for enhancements in both speed and graphical output.
  \item \emph{Safety:} Make sure the system is reliable, minimize memory leaks and undifined behaviours
  \item \emph{Cross-Platform:} Ensure the engine is not tied to a specific platform, operating system or graphics backend
  \item \emph{Futre-Proofing:} Build the engine on a forward-looking graphics API, designed to be efficient, powerful, and broadly supported.
\end{enumerate}

\section{Report structure}
The report is comprised of 5 parts:

\begin{itemize}
    \item \Cref{intro} gives an introduction to the project and the report
    \item \Cref{backgorund} details the project background, literature review and related work
    \item \Cref{methodology} presents a detailed walkthrough the designing and development of the project
    \item \Cref{results} shows the results and experiments of the project
    \item \Cref{conclusions} presents the project conclusions
\end{itemize}
